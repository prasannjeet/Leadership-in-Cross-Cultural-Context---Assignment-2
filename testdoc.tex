%!TEX TS-program = xelatex
%!TEX encoding = UTF-8 Unicode
%\documentclass{icisfinal}
\documentclass{article}

\usepackage{graphicx}
\usepackage{framed} %for adding a framebox
\usepackage[framemethod=tikz]{mdframed} %to change the background color in 'begin{framed}
\usepackage{booktabs}
\usepackage{pbox} %for using \pbox in the big table
\usepackage{float}
\usepackage[margin=1.25in]{geometry}
\usepackage{hyperref} %for colored links
\usepackage{parskip} % for disabling indentation in paragraphs
% \usepackage{lineno,hyperref}

\usepackage[
backend=biber,
style=chicago-authordate,
% citestyle=authoryear
]{biblatex}


\hypersetup{		% defining hyperlink colors
	colorlinks,
	linkcolor={red!50!black},
	citecolor={blue!80!black},
	urlcolor={blue!50!black},
}

\urlstyle{tt}


\def\changemargin#1#2{\list{}{\rightmargin#2\leftmargin#1}\item[]} 	%for paragraphs with...
\let\endchangemargin=\endlist

\title{Leadership in Cross Cultural Context Assignment 2 }
% \author{Prasannjeet Singh}
%\researchtype{Research in Progress}
%\shorttitle{This is my test title}
%\track{Human-computer Interaction}


\addbibresource{references.bib}
% \author{Prasannjeet Singh}

\usepackage{tabularx}  % To have the table fill out the page

\begin{document}

\maketitle

%\begin{center}
%    \LARGE \textbf{Developing transport performance measures for Construction Logistic Solutions}
%\end{center}

\begin{center}
    \large \textbf{Prasannjeet Singh}\\
    \small Växjö, Sweden
\end{center}

% \setlength{\parindent}{0pt}

\section*{Chapter 7: Negotiating global partnerships}
%\hline
\textbf{Key term: Collaboration (Conflict resolution)}
    
“Collaboration” is one of the five strategies (accommodation, competition, avoidance and compromise) for resolving conflicts. The choice of conflict resolution strategies depends upon two factors i.e. importance of a relationship and importance of outcome. Collaboration strategy is used when both factors are high. In other words, collaboration strategy refers to a joint search for a solution to a problem representing a win-win for all parties involved (\cite[p.~212]{steers2013management}). 

\textbf{Challenges and possibilities concerning multicultural wok environments}

The challenge lies in determining the right situation for this strategy as it is not always obvious that what approach a manager should take. Because this is hard to identify that how crucial a particular situation is for other party. “Power” among parties also play crucial role in approaching certain situations (\cite[p.~213]{steers2013management}). In addition to this, considering multi-culture work environment, different countries will approach same situation differently (\cite[p.~214]{steers2013management}) which pose a great challenge for a manager to resolve a conflict. As, both relation and outcome stakes are high so collaboration should be done by understanding all cultural assumptions of the parties involved.

\textbf{Strategies to managing multicultural work teams}

According to \cite{brett2009managing} there are \textbf{four} strategies to manage multicultural work teams and that is \textbf{Adaptation} (Acknowledging cultural gaps openly and working around them), \textbf{Structural intervention} (Changing shape of the team), \textbf{Managerial intervention} (setting norms early or bringing in a higher-level manager, \textbf{Exit} (removing a team member when other options have failed). 

For collaboration strategy to become effective, adaptation strategy will work well because this will give good understanding of parties involved both in terms of thinking and behaviour.



\section*{Chapter 8: Managing ethical conflicts}
%\hline
\textbf{Key term: Ethics}

The term “Ethics” is referred as the set of guiding principles regarding one on one relationships between individuals (\cite[p.~236]{steers2013management}).

\textbf{Challenges and possibilities concerning multicultural work environments}

The major challenge that managers in multi-culture work environment can face is that different cultures can have different ethical standard that may result in different perceptions and several judgements (\cite{dimitrova}). Different multinational companies have different views regarding child labour, gender discrimination, environmental pollution, nepotism, human rights, corruption etc. and even within a company, employee can have conflict with the company ethics. And as a result, it would become difficult for a manager to keep such employees motivated. Ethical understanding also make decision making challenging for the managers because there is possibility of many cross-cultural conflicts due to ethical misunderstandings (\cite{paul_et_al}).

\textbf{Strategies to managing multicultural work teams}

"Adaptation strategy" will best suit ethical consideration because it will allow managers to understand cross cultures in a better way and help them understand differences in ethical standards.

\section*{Chapter 9: Managing work and motivation}
%\hline
\textbf{Key term: Intrinsic rewards}

"Intrinsic rewards” are rewards that employee feel as a result of completing one´s job satisfactorily such as recognition or pat on back from boss, proud moment etc. (\cite[p.~275]{steers2013management}). 

\textbf{Challenges and possibilities concerning multicultural work environments}

Challenge within intrinsic reward is that, “reward preferences” vary from employee to employee such as some employee like to have recognition of his/her work in some way, whereas the other just want boss smile or thumbs up or pat on back etc. Reward preference are also culture bound such as some cultures value job security, some give importance to good interpersonal skills, some consider individual status and respect as the most important reward etc. In multicultural work environment, it is very important for a manager to know his/her employees in a better way and then reward employee’s effort in a way employee expects. If manager reward employee in other way as employee is expecting, then this will not motivate employee. Which means bigger responsibility lies on manager shoulders to know his/her employees well i.e. what kind of rewards motivate his/her subordinates? (\cite[p.~275]{steers2013management}).

\textbf{Strategies to managing multicultural work teams}

In this case, “adaptation” strategy will work because according to \cite{brett2009managing} this strategy leads to acknowledging cultural gaps openly and working around them. The more the manager or employees know each other´s culture, habits, wishes the more it would be easy to know intrinsic motivators of employees or co-workers. 


\section*{Chapter 10: Managing global teams}
%\hline
\textbf{Key term: Global teams}

“Global team” refers to a group of heterogeneous employees from two or more countries, and sometimes two or more companies, who work together to coordinate, develop, or manage some aspect of a firm's global operations. Global teams provide the skill set from different social, cultural and business aspects and play a major role in the success of goals and targets (\cite[p.~295]{steers2013management}).

\textbf{Challenges and possibilities concerning multicultural work environments}

Although “global teams” are more creative in developing ideas and solutions and results in increased understanding of global markets. But due to different working habits, conflicts and misunderstandings have more chance to occur, decisions take longer time to get implemented, even it takes a lot of time to finalize the decisions, bonding among employees is difficult to achieve so it takes a lot of time to know each other, thus resulting in more work for managers to manage the diverse work force (\cite[p.~296]{steers2013management}).   

\textbf{Strategies to managing multicultural work teams}

“Adaptation strategy” will fulfil the purpose for global teams as it emphasizes the importance of knowing each other's culture.

\section*{Chapter 11: Managing global assignments}
%\hline
\textbf{Key term: Psychological adjustment}

Psychological adjustment can be defined as the process of developing a way of life in the new country that is personally satisfying (\cite[p.~342]{steers2013management}). 


\textbf{Challenges and possibilities concerning multicultural work environments}

Adjusting oneself in a new environment can be stressful and tiring because of too much new information and need to learn that information as soon as possible. Employees also find it difficult to interpret new cues because they can not use their past experience to interpret and respond to cues. This makes new employees frustrated and depressed. New/ Foreign employees pass through different stages during this whole process of settling in a new environment i.e. honeymoon, disillusionment, adaptation, bi-culturalism (\cite[p.~345]{steers2013management}). For a manager, it is really hard to guess sometimes at what stage the new employee is and how much work load he/she can manage at a certain point in time. Too much workload can cause demotivation and new employees can feel loss of control whereas giving less work can cause boredom or lack of interest. The challenge for a manager is to balance each and every aspect of new employee work life. And also to know what the employee is feeling, what difficulties the employee is currently facing both in work and personal life etc.  

\textbf{Strategies to managing multicultural work teams}

“Adaptation strategy” will work in this scenario also because it allows co-workers and managers to know each other better. 

\section*{Overall Summary}

From all above discussion, it can be said that “culture” plays an important role in building good interpersonal relationships as well as increased performance and productivity. In today´s high tech business world, the skill that is most challenging is managerial/leadership skill because it has to deal with humans and human behaviour changes all the time. Humans do not behave or act the same even in similar situations. So, it takes a lot of wisdom, skill and empathy to manage employees. Adaptation strategy is a good strategy to manage multicultural work teams because it allows employees to know each other´s culture better and get adapted.  


% \bibliographystyle{misq}
% \bibliography{references}
\newpage
\printbibliography

\end{document}
%%% Local Variables:
%%% mode: latex
%%% TeX-master: t
%%% End:
